\newpage
\begin{center}
\large \bf
APLIKACJA DO ROZPOZNAWANIA EMOCJI W~SYGNALE MOWY
\end{center}

\section*{Streszczenie}
Celem pracy było stworzenie aplikacji klasyfikującej emocje na podstawie sygnału mowy.
Zawiera opis narzędzi wykorzystanych w~aplikacji oraz jej budowę.
W ostatniej części opisano bazę danych, zaimplementowane rozwiązanie oraz analizę wyników określających rozpoznawalność poszczególnych emocji spośród siedmiu klas, a~także parami.
Całą aplikację zaimplementowano w~języku Python z~wykorzystaniem pakietów Numpy, Scikit-learn i~Python speech features.

Praca została podzielona na 3 główne części, spośród których pierwsza opisuje metody i~narzędzia służące rozpoznawaniu emocji w~głosie.
Zawiera ona przybliżenie modelu emocji, ekstrakcji cech i~ich analizę statystyczną oraz redukcję wymiarowości przestrzeni w~której zostały opisane cechy.
Kolejna część przedstawia strukturę i~budowę aplikacji z~wyszczególnieniem istotnych fragmentów kodu.
Ostatnia część zawiera opis wykorzystanej bazy danych, wyekstrahowanych cech wypowiedzi oraz analizę wyników klasyfikacji emocji.
Badanie pokazuje skuteczność rozróżniania emocji spośród siedmiu klas oraz parami z~zastosowaniem różnych klasyfikatorów i~strategii redukcji wymiarowości.

\bigskip
{\noindent\bf Słowa kluczowe:} klasyfikacja, emocje, głos, mowa

\newpage

\begin{center}
\large \bf
THESIS TITLE
\end{center}

\section*{Abstract}
This thesis presents a~novel way of using a~novel algorithm to solve complex
problems of filter design. In the first chapter the fundamentals of filter design
are presented. The second chapter describes an original algorithm invented by the
authors. Is is based on evolution strategy, but uses an original method of filter
description similar to artificial neural network. In the third chapter the implementation
of the algorithm in C~programming language is presented. The fifth chapter contains results
of tests which prove high efficiency and enormous accuracy of the program. Finally some
posibilities of further development of the invented algoriths are proposed.

\bigskip
{\noindent\bf Keywords:} thesis, LaTeX, quality

\vfill
