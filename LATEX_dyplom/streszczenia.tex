\newpage
\begin{center}
\large \bf
Aplikacja do rozpoznawania emocji w sygnale mowy
\end{center}

\section*{Streszczenie}
Emocje odgrywają elementarną rolę w komunikacji międzyludzkiej.
Prawidłowa interpretacja emocji jest konieczna do prawidłowej komunikacji człowiek-człowiek i człowiek-maszyna.
Można wyróżnić 6 emocji podstawowych, których kombinacje tworzą bardziej złożone emocje.
Każda emocja znacząco wpływa na organizm człowieka, jego zdolność percepcji i działania.
Wykorzystując narzędzia cyfrowego przetwarzania sygnałów można stworzyć opis liczbowy
sygnału mowy stanowiący zestaw cech bazowych.
Wypowiedzi zostały podzielone na równe ramki zawierające krótkie jej fragmenty.
Dla każdej ramki sygnału wypowiedzi zostały wyznaczone cechy bazowe.
Opis statystyczny cech bazowych wszystkich ramek wypowiedzi tworzy zestaw cech właściwych wykorzystanych w klasyfikacji.
Ograniczono liczbę używanych cech stosując metody redukcji wymiarowości i selekcji cech.
Opisano klasyfikatory, które zostały użyte w badaniu.
Rozdział trzeci opisuje platformę i biblioteki które zostały zastosowane w aplikacji.
Pokazana została struktura aplikacji oraz implementacja najistotniejszych jej modułów i komunikacji między modułami.
Badanie rozpoczyna się opisem użytej bazy danych wypowiedzi emocjonalnych.
Wymienione zostały cechy użyte w klasyfikacji.
Wyniki badań klasyfikacji emocji w głosie dla dwóch i siedmiu emocji zostały przedstawione w postaci tabel.

\bigskip
{\noindent\bf Słowa kluczowe:} klasyfikacja, emocje, głos, mowa

\vskip 2cm


\begin{center}
\large \bf
THESIS TITLE
\end{center}

\section*{Abstract}
This thesis presents a novel way of using a novel algorithm to solve complex
problems of filter design. In the first chapter the fundamentals of filter design
are presented. The second chapter describes an original algorithm invented by the
authors. Is is based on evolution strategy, but uses an original method of filter
description similar to artificial neural network. In the third chapter the implementation
of the algorithm in C programming language is presented. The fifth chapter contains results
of tests which prove high efficiency and enormous accuracy of the program. Finally some
posibilities of further development of the invented algoriths are proposed.

\bigskip
{\noindent\bf Keywords:} thesis, LaTeX, quality

\vfill
