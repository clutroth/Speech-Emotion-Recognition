\newpage
\begin{center}
\large \bf
APLIKACJA DO ROZPOZNAWANIA EMOCJI W~SYGNALE MOWY
\end{center}

\section*{Streszczenie}
Celem pracy było stworzenie aplikacji do rozpoznawania emocji na podstawie sygnału mowy.
Praca została podzielona na 3 główne części, spośród których pierwsza opisuje metody i~narzędzia służące rozpoznawaniu emocji w~głosie.
Zawiera ona przybliżenie modelu emocji, ekstrakcji cech i~ich analizę statystyczną, redukcję wymiarowości przestrzeni, w~której zostały opisane cechy oraz klasyfikację emocji.
Kolejna część przedstawia strukturę i~budowę aplikacji z~wyszczególnieniem istotnych fragmentów kodu
oraz opis wykorzystanych narzędzi.
Aplikacja została napisana w języku Python z wykorzystaniem bibliotek numpy, skit-learn i python-speach-features.
Ostatnia część pracy zawiera opis wykorzystanej bazy danych, wyekstrahowanych cech wypowiedzi oraz analizę wyników klasyfikacji emocji.
Badanie pokazuje skuteczność rozróżniania emocji spośród siedmiu klas oraz parami z~zastosowaniem różnych klasyfikatorów i~strategii redukcji wymiarowości.
Wykorzystano algorytmy rekurencyjnej eliminacji cech i analizę składowych głównych w celu zmniejszenia liczby rozpatrywanych cech.
Do klasyfikacji użyto klasyfikatora najbliższych sąsiadów, maszyny wektorów wspierających i sztuczne sieci neuronowe.

\bigskip
{\noindent\bf Słowa kluczowe:} klasyfikacja, emocje, głos, mowa

\newpage

\begin{center}
\large \bf
APPLICATION FOR EMOTION RECOGNITION BASED ON SPEECH SIGNAL
\end{center}

\section*{Abstract}
This thesis purpose was creating an application for emotion recognition based on speech signal.
Thesis is composed of three chapters.
First chapter describes methods and tools used in speech emotion recognition.
It includes emotion model introduction, features extraction, statistical analysis, and dimension reduction techniques.
Next chapter contains application architecture with important code listings exposed, and used tools description.
Application uses numpy, skit-learn, and python-speach-features libraries.
Last chapter includes used database account, extracted speech features, and emotion classification results analysis.
Research shows classification accuracy in case for tow and seven classes with different classifiers, and dimension reduction strategies.
Recursion feature elimination, and principal component analysis have been used for dimension reduction.
Classification have been done with nearest neighbours classifier, support vector machines, nad artificial neural networks. 

\bigskip
{\noindent\bf Keywords:} classification, emotion, speech, voice

\vfill
