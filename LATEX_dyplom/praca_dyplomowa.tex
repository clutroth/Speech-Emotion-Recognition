\documentclass[a4paper,12pt,twoside,openany]{report}
%
% Wzorzec pracy dyplomowej
% J. Starzynski (jstar@iem.pw.edu.pl) na podstawie pracy dyplomowej
% mgr. inż. Błażeja Wincenciaka
% Wersja 0.1 - 8 października 2016
%
\usepackage{polski}
\usepackage{helvet}
\usepackage[T1]{fontenc}
\usepackage{anyfontsize}
\usepackage[utf8]{inputenc}
\usepackage[pdftex]{graphicx}
\usepackage{tabularx}
\usepackage{array}
\usepackage[polish]{babel}
\usepackage{subfigure}
\usepackage{amsfonts}
\usepackage{verbatim}
\usepackage{indentfirst}
\usepackage[pdftex]{hyperref}


% rozmaite polecenia pomocnicze
% gdzie rysunki?
\graphicspath{ {rys/} }
% oznaczenie rzeczy do zrobienia/poprawienia
\newcommand{\TODO}{\textbf{TODO}}


% wyroznienie slow kluczowych
\newcommand{\tech}{\texttt}

% na oprawe (1.0cm - 0.7cm)*2 = 0.6cm
% na oprawe (1.1cm - 0.7cm)*2 = 0.8cm
%  oddsidemargin lewy margines na nieparzystych stronach
% evensidemargin lewy margines na parzystych stronach
\def\oprawa{1.05cm}
\addtolength{\oddsidemargin}{\oprawa}
\addtolength{\evensidemargin}{-\oprawa}

% table span multirows
\usepackage{multirow}
\usepackage{enumitem}    % enumitem.pdf
\setlist{listparindent=\parindent, parsep=\parskip} % potrzebuje enumitem

%%%%%%%%%%%%%%% Dodatkowe Pakiety %%%%%%%%%%%%%%%%%
\usepackage{prmag2017}   % definiuje komendy opieku,nrindeksu, rodzaj pracy, ...
\usepackage{xspace}

%%%%%%%%%%%%%%% Strona Tytułowa %%%%%%%%%%%%%%%%%
% To trzeba wypelnic swoimi danymi
\title{Aplikacja do rozpoznawania emocji w sygnale mowy}

% autor
\author{Wojciech Decker}
\nrindeksu{252545}


\opiekun{dr inż. Andrzej Majkowski}
\konsultant{prof. Dzielny Konsultant}  % opcjonalnie
\terminwykonania{1 września 2017} % data na oświadczeniu o samodzielności
\rok{2017}


% Podziekowanie - opcjonalne
\podziekowania{\noindent
{\Large Podziękowania}
\bigskip

Chciałbym bardzo serdecznie podziękować dr inż. Andrzejowi Majkowskiemu za pomoc merytoryczną i cierpliwe wskazywanie błędów. Nie tylko tych rzeczowych.
Dziękuję również Maćkowi, Michałowi i Bartkowi, którzy zawsze mnie wysłuchiwali, służyli radą i byli wyrozumiali, kiedy pisanie tej pracy stawiałem ponad inne obowiązki.

\bigskip

{\raggedleft
Wojciech Decker
}

}

% To sa domyslne wartosci
% - mozna je zmienic, jesli praca jest pisana gdzie indziej niz w ZETiIS
% - mozna je wyrzucic jesli praca jest pisana w ZETiIS
%\miasto{Warszawa}
%\uczelnia{POLITECHNIKA WARSZAWSKA}
%\wydzial{WYDZIAŁ ELEKTRYCZNY}
%\instytut{INSTYTUT ELEKTROTECHNIKI TEORETYCZNEJ\linebreak[1] I~SYSTEMÓW INFORMACYJNO-POMIAROWYCH}
\zaklad{ZAKŁAD SYSTEMÓW INFORMACYJNO POMIAROWYCH}
%\kierunekstudiow{INFORMATYKA}

% domyslnie praca jest inzynierska, ale po odkomentowaniu ponizszej linii zrobi sie magisterska
\pracamagisterska
%%% koniec od P.W

\opinie{%
	\input{opiniaopiekuna.tex}
	\newpage
	\input{recenzja.tex}
}

\streszczenia{
	\newpage
\begin{center}
\large \bf
APLIKACJA DO ROZPOZNAWANIA EMOCJI W~SYGNALE MOWY
\end{center}

\section*{Streszczenie}
Celem pracy było stworzenie aplikacji do rozpoznawania emocji na podstawie sygnału mowy.
Praca została podzielona na 3 główne części, spośród których pierwsza opisuje metody i~narzędzia służące rozpoznawaniu emocji w~głosie.
Zawiera ona przybliżenie modelu emocji, ekstrakcji cech i~ich analizę statystyczną, redukcję wymiarowości przestrzeni, w~której zostały opisane cechy oraz klasyfikację emocji.
Kolejna część przedstawia strukturę i~budowę aplikacji z~wyszczególnieniem istotnych fragmentów kodu
oraz opis wykorzystanych narzędzi.
Aplikacja została napisana w języku Python z wykorzystaniem bibliotek numpy, skit-learn i python-speach-features.
Ostatnia część pracy zawiera opis wykorzystanej bazy danych, wyekstrahowanych cech wypowiedzi oraz analizę wyników klasyfikacji emocji.
Badanie pokazuje skuteczność rozróżniania emocji spośród siedmiu klas oraz parami z~zastosowaniem różnych klasyfikatorów i~strategii redukcji wymiarowości.
Wykorzystano algorytmy rekurencyjnej eliminacji cech i analizę składowych głównych w celu zmniejszenia liczby rozpatrywanych cech.
Do klasyfikacji użyto klasyfikatora najbliższych sąsiadów, maszyny wektorów wspierających i sztuczne sieci neuronowe.

\bigskip
{\noindent\bf Słowa kluczowe:} klasyfikacja, emocje, głos, mowa

\newpage

\begin{center}
\large \bf
APPLICATION FOR EMOTION RECOGNITION BASED ON SPEECH SIGNAL
\end{center}

\section*{Abstract}
This thesis purpose was creating an application for emotion recognition based on speech signal.
Thesis is composed of three chapters.
First chapter describes methods and tools used in speech emotion recognition.
It includes emotion model introduction, features extraction, statistical analysis, and dimension reduction techniques.
Next chapter contains application architecture with important code listings exposed, and used tools description.
Application uses numpy, skit-learn, and python-speach-features libraries.
Last chapter includes used database account, extracted speech features, and emotion classification results analysis.
Research shows classification accuracy in case for tow and seven classes with different classifiers, and dimension reduction strategies.
Recursion feature elimination, and principal component analysis have been used for dimension reduction.
Classification have been done with nearest neighbours classifier, support vector machines, nad artificial neural networks. 

\bigskip
{\noindent\bf Keywords:} classification, emotion, speech, voice

\vfill

}
\newcommand{\ang}[1]{\textit{(ang. #1)}}
\newcommand{\MATLAB}{\textsc{Matlab}\xspace}

\begin{document}
\maketitle
%-----------------
% Wstęp
%-----------------
\chapter{Wstęp}
\label{ch:wstep}
% krótka definicja emocji
Emocje to stany ludzkiego umysłu.
Powstają w odpowiedzi na zdarzenie, są ukierunkowane i krótkotrwałe.
Różnią się intensywnością i zabarwieniem.
Wpływają na interpretację bodźców z otoczenia,
myśli a w konsekwencji mają istotny wpływ na zachowanie i reakcje.

% informacje zakodowane w mowie
Mowa jest nośnikiem informacji wykorzystywanym w komunikacji międzyludzkiej,
oraz pomiędzy człowiekiem i komputerem \ang{Human-Computer Interaction}.
Komunikat głosowy składa się z treści językowej,
którą można zapisać w formie tekstu,
oraz akustycznej, która również opisuje wypowiedź takie jak:
drżący ze strachu głos, czy ciężki oddech świadczący o gniewie.

% Zastosowanie maszynowego rozpoznawania emocji
Rozpoznawanie emocji mówcy jest istotne w aplikacjach wykorzystujących mowę w komunikacji człowiek-maszyna.
Zwłaszcza, jeśli odpowiedź systemu jest uzależniona od nastroju człowieka.
Klasyfikacja emocji wypowiedzi jest wykorzystywana w terapiach,
gdzie terapeuta wspiera się maszyną w odczytywaniu emocji pacjenta.
Systemy tłumaczenia maszynowego mowy mogą wykorzystać informacje mówiące o kontekście wypowiedzi i stanie mówcy.
Telefoniczne centra obsługi klienta \ang{call center} rozpoznają stan klienta,
oraz jego reakcję na ofertę prezentowaną przez konsultanta.

% Jaki jest cel?
Celem pracy jest przybliżenie tematyki komputerowego rozpoznawania mowy,
przegląd wykorzystywanych narzędzi, oraz ocena znanych rozwiązań.
Produktem końcowym będzie aplikacja komputerowa rozpoznająca emocje w sygnale mowy.
W trakcje tworzenia pracy zostanie dokonana analiza znanych rozwiązań tego zagadnienia.
Następnie zostanie opracowany schemat aplikacji, bazujący na poprzednich badaniach,
pozwalający stworzyć aplikację w warunkach powstawania pracy magisterskiej.
Kolejnym etapem będzie implementacja aplikacji w środowisku MATLAB.
Na zakończenie zostaną przeprowadzone testy aplikacji,
analiza wyników i prezentacja wniosków.

\chapter{Opis prac}
\section{Opis korpusu}
W pracy wykorzystano korpus emocji w mowie powstały w Akademii Górniczo-Hutniczej w Krakowie.
Nagrania zawierają pięć emocji podstawowych: radość, smutek, złość, strach, zdziwienie.
Ponadto zostały nagrane wypowiedzi w tonie neutralnym jako punkt odniesienia i ironicznym, jako emocja złożona.
W nagraniu wzięło udział 6 mężczyzn i 6 kobiet od 20 do 30 lat. 
Mówcy byli zarówno profesjonalnymi aktorami jak i amatorami, czy wolontariuszami.
Baza została zrealizowana jako korpus emocji odgrywanych. 

Nagrano 4 typy wypowiedzi.
\begin{description}
	\item [Zdania] będące sekwencją 46 prostych wypowiedzi często wykorzystywanych w życiu codziennym 
		np. ,,Dzień dobry'', ,,Witam serdecznie''. 
	\item [Polecenia]  np. ,,Nowy'', ,,Otwórz''. 
		Jest to głosowa reprezentacja standardowych komend wykorzystywanych w komunikacji człowieka z komputerem.
	\item [Cyfry] od 0 do 9.
	\item [Tekst] czyli fragment artykułu. 
		Lity tekst będący spójnym logicznie ciągiem zdań, lecz wyrwanym z kontekstu.
\end{description}
Dla każdego z 6 mówców zarejestrowano każdy typ wypowiedzi we wszystkich stanach emocjonalnych.

Korpus składa się z opisu, oraz wypowiedzi posegregowanych w katalogach oznaczających odgrywaną emocję.
Nazwy plików wskazują na mówcę, emocję oraz typ wypowiedzi.
Nagrania zostały zapisane w formie plików WAV przy częstotliwości próbkowania 44100Hz i rozdzielczości 16 bit ~\cite{Igras2009}.

\section{Przetwarzanie wstępne}
Większość nagrań nie została przetworzona.
Sygnał niektórych nagrań został poddany standaryzacji.
\begin{figure}[h]
	\centering
	\includegraphics[width=\textwidth]{MGR_ZL_C-plot}
	\caption{Przykład ustandaryzowanego sygnału}
\end{figure}
\begin{figure}[h]
	\centering
	\includegraphics[width=\textwidth]{MGR_ZD_T-plot}
	\caption{Przykład nie ustandaryzowanego sygnału}
\end{figure}

Nagrania z korpusu nie były przetworzone wstępnie.
Poziom głośności nagrań różnił się pomiędzy nagraniami.
Zapis wypowiedzi został ustandaryzowany w celu zniwelowania różnic wynikających z głośności nagrania.
\section{}
\bibliography{praca_dyplomowa}{}
\bibliographystyle{plain}
\end{document}
%+++ END +++
